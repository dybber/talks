\documentclass{beamer}
\usepackage[utf8]{inputenc}

\usepackage{semantic}
\usepackage{graphicx}
\usepackage{booktabs}
\usepackage{todonotes}
\presetkeys{todonotes}{inline}{}

\usepackage[absolute]{textpos}
% This is to get textpos to play well,
% A stackoverflow user mentioned that this could
% cause problems in some Acrobat Reader versions
% I'll take the chance
\usebackgroundtemplate{}


\mode<presentation> {
\usetheme{boxes} % When headline is wanted use Dresden theme instead
\usecolortheme{seagull}
\setbeamertemplate{footline}[page number]
\setbeamertemplate{navigation symbols}{}
}


%----------------------------------------------------------------------------------------
%	TITLE PAGE
%----------------------------------------------------------------------------------------

\title[APL \& TAIL] % bottom of every slide
  {Teaching kids programming, IT-creativity and modern tech} % title page

\author{\footnotesize{Martin Dybdal} \\ \footnotesize{\texttt{dybber@dybber.dk}}}

\institute {
DIKU \\
University of Copenhagen
}

\date{\footnotesize{Chalmers, 10 October 2015}}

\date{10 November 2015}

\begin{document}

{
\setbeamertemplate{headline}{}
\begin{frame}
  \begin{center}
    \includegraphics[width=0.7\textwidth]{imagery/codingpirates.png}
  \end{center}
\vspace{-1cm}
\titlepage
\end{frame}
}

%----------------------------------------------------------------------------------------
%	TABLE OF CONTENTS
%----------------------------------------------------------------------------------------

\begin{frame}
\frametitle{Overview}
\tableofcontents
\end{frame}

%----------------------------------------------------------------------------------------
%	CONTENT
%----------------------------------------------------------------------------------------

\section{What is Coding Pirates?}
\begin{frame}
\frametitle{What is Coding Pirates?}

\begin{textblock}{20}(5,5)
 \includegraphics[width=\textwidth]{imagery/cpthack-and-miss1337.png}
\end{textblock}

\begin{itemize}
\item Activity for kids aged 7-17 years
\item A playground - not just two more hours of school
\item An attempt to change the school system
\end{itemize}
\end{frame}

\subsection{Who are Coding Pirates?}
\begin{frame}
\frametitle{Who are Coding Pirates?}
\begin{itemize}
\item Non-profit organisation
\item +250 volunteers at Coding Pirates network
\item $\sim$700 paying members
\item 24 hubs in Denmark
\item additional 6 hubs from January 2016
\end{itemize}

\todo{Image with kids!}

\end{frame}

\begin{frame}
  \frametitle{Partners}

\begin{itemize}
\item Center for Teaching Development and Digital Media, Aarhus
  University
\item Department of Computer Science, University of Copenhagen
\item The libraries
\item Microsoft
\item The Danish IT Industri Association
  \begin{itemize}
  \item Computerworld
  \item Canon
  \item CapGemini
  \item NNIT
  \item \ldots
  \end{itemize}
\end{itemize}
\end{frame}

\subsection{The Coding Pirates philosophy}
\begin{frame}
\frametitle{The Coding Pirates philosophy}

\begin{quotation}
  "The problems now faced by mankind are largely due to man's own
  creativeness. Creativeness will need to account for much more if
  present problems are to be transcended with solutions".

  - Preface of "Explorations in Creativity", Editors: Ross L. Mooney,
  Taher A. Razik
\end{quotation}


Manifesto: http://codingpirates.dk/manifesto/
\end{frame}


\section{Coding Pirates in practice}
\begin{frame}
\frametitle{A normal day at Coding Pirates}

\end{frame}


\subsection{Scratch}
\begin{frame}
\frametitle{Scratch}

\end{frame}

\begin{frame}
\frametitle{Scratch 4 Arduino}

\end{frame}

% \section{Processing(.js)}
% \begin{frame}
% \frametitle{Processing(.js)}

% \end{frame}

\subsection{... and other technologies}
\begin{frame}

\frametitle{Other technologies used}
\todo{find logos}

\begin{itemize}
\item Processing(.js) via KhanAcademy
\item Arduino
\item Blender (3D modelling)
\item Unity (2D and 3D games)
\item LEGO Mindstorms and LEGO WeDo (expensive)
\item littleBits (expensive)
\end{itemize}
\end{frame}

\subsection{Showcase of projects}

\subsection{Difficulties ahead}
\begin{frame}
\frametitle{Difficulties identified}
\begin{itemize}
\item Few from age 14 and up
\item Few girls
\item Fostering friendships is hard, but important
\item Further education of volunteers
\end{itemize}
\end{frame}

\begin{frame}
  \frametitle{Age and gender (including waiting list)}
  \centerline{\includegraphics[width=\textwidth]{../datacrunching/age-gender-hist}}
\end{frame}

\subsection{The Coding Pirates Community}
\begin{frame}
  \frametitle{Volunteer community}
  \includegraphics[width=\textwidth]{imagery/unity-for-volunteers.jpg}
\end{frame}


\section{Computing in schools}
\begin{frame}
  \frametitle{Computing in schools}
  \begin{quotation}
    "The computer is the Proteus of machines. Its essence is its
    universality, its power to simulate. Because it can take on a
    thousand forms and can serve a thousand functions, it can appeal
    to a thousand tastes."

    - Seymour Papert, in Mindstorms
  \end{quotation}

  We can already use this when teaching e.g. history, biology, chemistry,
  or language classes!

  \begin{itemize}
  \item Make a game that teaches grade $N-1$ about photosynthesis
  \item Make a game that teaches grade $N-1$ about life in ancient Rome
  \item Make an interactive story that tells the story XYZ
  \end{itemize}

  More fun than a poster or a written report!
\end{frame}

\begin{frame}
  \frametitle{But what do we want schools to teach?}

  \begin{itemize}
  \item Teach computing as a discipline, e.g. like math
    \begin{itemize}
    \item Algorithms vs. data
    \item Systematic problem solving
    \item Computational thinking
    \end{itemize}
  \item Teach computing as a craft/skill, e.g. like woodwork
    \begin{itemize}
    \item Focus on creation and tools
    \item Creative and reflective thinking
    \end{itemize}
  \item As a separate discipline or inside other classes?
  \end{itemize}
\end{frame}

\section{DIKU's involvement}
\begin{frame}
\frametitle{Why does a university use time teaching tweens and teens?}
\begin{itemize}
\item The teachers needs our expertise
\item Defining how computing should be taught in Schools
\item Potential research areas
\item Teacher education and re-education
\item Because we have connections to potential volunteers (e.g. alumni)
\item Good publicity and great advertisement
\end{itemize}
\end{frame}

\begin{frame}
  \frametitle{Links}
  \begin{itemize}
  \item Coding Pirates website: http://codingpirates.dk
  \item Manifesto: http://codingpirates.dk/manifesto/
  \end{itemize}
\end{frame}

\begin{frame}
  \frametitle{Coding Pirates in Gothenburg?}
  \begin{center}
    \includegraphics[origin=c,angle=-20,width=0.7\textwidth]{imagery/wanted}
  \end{center}

\end{frame}


\end{document}