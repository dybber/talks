\documentclass{beamer}
\usepackage[utf8]{inputenc}

\usepackage{semantic}
\usepackage{graphicx}
\usepackage{booktabs}
%\usepackage{todonotes}
\usepackage[absolute,overlay]{textpos}
\usepackage{tikz}
\usetikzlibrary{decorations.pathmorphing}

\mode<presentation> {
\usetheme{boxes} % When headline is wanted use Dresden theme instead
\usecolortheme{seagull}
\setbeamertemplate{footline}[page number]
\setbeamertemplate{navigation symbols}{}
}

%\input{apl.tex}

%----------------------------------------------------------------------------------------
%	TITLE PAGE
%----------------------------------------------------------------------------------------

\title[FCL] % bottom of every slide
  {Introduction to GPU programming} % title page


\author{\footnotesize{Martin Dybdal} \\ \footnotesize{\texttt{dybber@dybber.dk}}}

\institute {
HIPERFIT research center \\
DIKU \\
University of Copenhagen
}

\date{\footnotesize{Workshop on Presentation techniques, 29 March 2016}}

% \title[Group Theory]{A longer title of the talk concerning Group Theory}
% \author[short name]{My full name}
% \institute[My Inst.]{Full Institut Name}
% % logo of my university

\date[4 March 2016]{4 March 2016}

\begin{document}

{
\setbeamertemplate{headline}{}
\begin{frame}
  \titlepage
\end{frame}
}

%----------------------------------------------------------------------------------------
%	TABLE OF CONTENTS
%----------------------------------------------------------------------------------------

% \begin{frame}
% \frametitle{Overview}
% \tableofcontents
% \end{frame}

%----------------------------------------------------------------------------------------
%	CONTENT
%----------------------------------------------------------------------------------------

\section{Motivation}

\begin{frame}
\frametitle{Physics simulation}
\includegraphics[width=\textwidth]{graphics/opencl_nbody}
\end{frame}

\begin{frame}
\frametitle{Financial simulation}
\includegraphics[width=\textwidth]{graphics/lsmplot}
\end{frame}

\tikzset{snake arrow/.style=
{-stealth,
decorate,
decoration={snake,amplitude=.4mm,segment length=2mm,post length=0.9mm}},
}

\section{CPU vs. GPU}

\begin{frame}[fragile]
\frametitle{Standard CPU}
\only<1-3>{
\begin{tikzpicture}
\begin{scope}[every edge/.append style = {snake arrow}]
\node  at (0,2.5) {\footnotesize Core 0}; 
\node  at (1.25,2.5) {\footnotesize Core 1}; 
\node  at (2.5,2.5) {\footnotesize Core 2}; 
\node  at (3.75,2.5) {\footnotesize Core 3}; 
\draw (0,2) edge (0,0)
      (1.25,2) edge (1.25,0)
      (2.5,2) edge (2.5,0)
      (3.75,2) edge (3.75,0);
\end{scope}
\end{tikzpicture}
}
\only<4->{
\begin{tikzpicture}
\begin{scope}[every edge/.append style = {snake arrow}]
\node  at (0,2.5) {\footnotesize Core 0}; 
\node  at (1.25,2.5) {\footnotesize Core 1}; 
\node  at (2.5,2.5) {\footnotesize Core 2}; 
\node  at (3.75,2.5) {\footnotesize Core 3}; 
\draw (0,1) edge (0,0)
      (0,2) edge (0,1)
      (1.25,0.66) edge (1.25,0)
      (1.25,1.33) edge (1.25,0.66)
      (1.25,2) edge (1.25,1.33)
      (2.5,2) edge (2.5,0)
      (3.75,2) edge (3.75,0);
\end{scope}
\end{tikzpicture}
}
\only<5->{
\begin{textblock*}{6cm}(6.5cm,2cm)
\includegraphics[width=\textwidth]{graphics/ivybridge.jpg}
\end{textblock*}
}
\vspace{3mm}
\begin{itemize}
\item<2-> One operation at a time
\item<3-> Few compute units (cores)
\item<4-> Fast at switching between tasks
\item<5-> Most transistors used for ``recalling''
\end{itemize}

\end{frame}

\begin{frame}[fragile]
\frametitle{GPU}
\begin{tikzpicture}
\begin{scope}[every edge/.append style = {snake arrow}]
\foreach \x in {0,...,16}
  \draw (0.25*\x,1) edge (0.25*\x,0);
\end{scope}
\foreach \y in {0,...,4}
  \draw [color=lightgray](0,0.2+0.2*\y) edge (4,0.2+0.2*\y);
%\draw [color=gray,thick](-0.25,-0.25) rectangle (4.25,1.25);
\end{tikzpicture}

% \vspace{2mm}
% \begin{tikzpicture}
% \begin{scope}[every edge/.append style = {snake arrow}]
% \foreach \x in {0,...,16}
%   \draw (0.25*\x,1) edge (0.25*\x,0);
% \end{scope}
% \foreach \y in {0,...,4}
%   \draw [color=lightgray](0,0.2+0.2*\y) edge (4,0.2+0.2*\y);
% %\draw [color=gray,thick](-0.25,-0.25) rectangle (4.25,1.25);
% \end{tikzpicture}

\vspace{10mm}

\begin{textblock*}{4cm}(8.2cm,1cm)
\includegraphics[width=\textwidth]{graphics/gtx980}
\end{textblock*}

\begin{itemize}
\item<2-> Identical operations on diff. data
\item<3-> Thousands of compute units (cores)
\item<4-> Tasks executed in order (queued)
\item<5-> Most transistors used for computing
\end{itemize}

\end{frame}

\begin{frame}[fragile]
\frametitle{CPU vs. GPU programming}

\begin{textblock*}{0.6\textwidth}(0.1\textwidth,2cm)
CPU programming
\begin{verbatim}
5+9
      14
14+3
      17
17+22
      39
\end{verbatim}
\end{textblock*}

\begin{textblock*}{0.7\textwidth}(0.5\textwidth, 2cm)
GPU programming
\begin{verbatim}
(2 4 6 8 10) + 100
      102 104 106 108 110

(102 104 106 108 110) * 2
      204 208 212 216 220
\end{verbatim}
\end{textblock*}
\end{frame}

\begin{frame}[fragile]
\frametitle{GPU programming}

Problem:
\begin{itemize}
\item GPU cores are bad at ``recalling''
\item manual control of ``scratch pad''
\end{itemize}

\end{frame}

\begin{frame}[fragile]
\frametitle{Fusion}

\begin{verbatim}
((2 4 6 8 10) + 100) * 2
      204 208 212 216 220
\end{verbatim}
\end{frame}

\begin{frame}
\frametitle{Summary}
\begin{itemize}
\item GPUs require many similar computations on different data
\item GPUs require attention to memory transactions (fusion)
\item GPU programming: as hard as programming CPUs in the 60s/70s
\end{itemize}
\end{frame}


\end{document}